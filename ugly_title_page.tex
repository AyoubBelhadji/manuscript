
%\newcommand*{\titleGM}{
%\makeatletter
%\def\@maketitle{%
\newcommand{\maketitleGM}{
%\pagestyle{empty}
    \begingroup % Create the command for including the title page in the document
    \thispagestyle{empty}
    \newgeometry{textwidth=15cm, top=0cm, bottom=1.5cm}

    \hbox{ % Horizontal box
        \hspace*{0.025\textwidth} % Whitespace to the left of the title page
        %\rule{2pt}{\textheight} % Vertical line
        %\hspace*{0.05\textwidth} % Whitespace between the vertical line and title page text
        \parbox[c]{\textwidth}{ % Paragraph box which restricts text to less than the width of the page
        \vspace*{-2em}
        {Numéro d'ordre: ??}
        {\LARGE \begin{center} Centrale Lille \end{center} }

        \vspace{2em}

        {\huge \begin{center} Thèse de Doctorat \end{center} }
        % {\Large \begin{center} En \end{center} }
        {\Large \begin{center}
        % Spécialité :
        \textbf{En Automatique, Génie informatique, Traitement du signal et image} \end{center} }

        \vspace{2em}

        \begin{center}
        	\Large Par \huge \textsc{Ayoub Belhadji} % Author name
        \end{center}

        \vspace{2em}

        {\Large
        \begin{center}
            Doctorat délivré par Centrale Lille
        \end{center} }

        \vspace{2em}

        {\begin{center}
        	{\noindent\LARGE \textbf{Subspace sampling using determinantal point processes.}}\\[1\baselineskip]
        	{\noindent\large \emph{Echantillonnage des sous-espaces à l'aide des processus ponctuels déterminantaux}}\\[1\baselineskip]
        \end{center}}

        \vspace{2em}

        {\begin{minipage}[b]{0.55\linewidth}
          \hspace{-2.5em}
          % \large Soutenue le 03 novembre 2020 devant le jury d'examen:
        \end{minipage}
        \hfill
        \begin{minipage}[b]{0.49\linewidth}

        \end{minipage}}

        \vspace{0.5em}

        {\hspace{-3em}
        \begin{tabular}{lll}
        % \textit{Président}
        %     &
        %     Pierre-Olivier {\scshape Amblard}
        %     &
        %     Directeur de Recherche CNRS, Université de Grenoble-Alpes
        %     \\
        \textit{Rapporteurs}
            &   Agnès {\scshape Desolneux }
            &   Directrice de Recherche CNRS, ENS Paris-Saclay
            \\
            &   -
            &   -\\
        \textit{Examinateurs}
            &   -
            &   -
            \\
            &   -
            &   -
            \\
            &   -
            &   -
            \\
        \textit{Directeurs de thèse}
            &   Pierre {\scshape Chainais}
            &   Professeur des universités, Centrale Lille
            \\
            &   Rémi {\scshape Bardenet}
            &   Chargé de Recherche CNRS, Université de Lille
        \end{tabular}
        }
        % \end{center}

        \vspace{2em}

        \begin{center}
            {\large Thèse préparée dans le Laboratoire} \\
            {\large \textit{Centre de Recherche en Informatique Signal et Automatique de Lille}\\ Université de Lille, Centrale Lille, CNRS, UMR 9189 - CRIStAL \\
            École Doctorale SPI 072}
        \end{center}

        \vspace{2em}

        % \begin{center}
        %     \includegraphics[height=1.3cm]{images/logo_centrale.png}\hspace{1em}
        %     \includegraphics[height=1.3cm]{images/logo_univ_lille.pdf}\hspace{1em}
        %     \includegraphics[height=1.3cm]{images/logo_CRIStAL.jpg}\hspace{1em}
        %     \includegraphics[height=1.3cm]{images/logo_CNRS.jpg}\hspace{1em}
        %     \includegraphics[height=1.3cm]{images/logo_INRIA.jpg}
        % \end{center}
        }
    }
    \endgroup
}
\makeatother
