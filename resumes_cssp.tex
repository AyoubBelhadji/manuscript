
\documentclass[twoside,11pt]{article}
%\usepackage{classicthesis}

\usepackage[margin=3cm]{geometry}

\usepackage[T1]{fontenc}
\usepackage{relsize}

\usepackage{amsmath}
\usepackage{amssymb}
\usepackage{bm}
\usepackage{bbm}
\usepackage{natbib}
\usepackage{bibentry}
\usepackage{cancel}


\usepackage{appendix}
%\usepackage{chngcntr}
\usepackage{etoolbox}
%\usepackage{subcaption}

\def\twofig{.49\textwidth}


%\usepackage[style=numeric,natbib=true]{biblatex}

\usepackage{myAlgorithm}

\setcounter{MaxMatrixCols}{20}

\usepackage[dvipsnames]{xcolor}
\definecolor{darkgreen}{rgb}{0,.5,.2}
% \newcommand{\rev}[1]{\textcolor{darkgreen}{#1}}
\newcommand{\rev}[1]{\textcolor{black}{#1}}


\bibliographystyle{chicago}

% \RequirePackage[colorlinks=true,citecolor=blue,allbordercolors={1 1 1}]{hyperref}



%\addbibresource{bibliography.bib}
%\bibliography{bibliography}
\usepackage{todonotes} 

\usepackage{subfig}

\usetikzlibrary{bayesnet,calc,angles,quotes}
\usepackage{tikz-3dplot}
\usetikzlibrary{decorations.pathreplacing}
\usepackage[utf8]{inputenc}

\usepackage{amsthm}
\newtheorem{corollary}{Corollary}
\newtheorem{theorem}{Theorem}
\newtheorem{proposition}{Proposition}
\newtheorem{claim}{Claim}
\newtheorem{definition}{Definition}
\newtheorem{lemma}{Lemma}
\newtheorem{assumption}{Assumption}
\newtheorem{problem}{Problem}
\newtheorem{example}{Example}

%\newtheorem{assumption}{Assumption}
%\renewcommand*{\theassumption}{\Alph{assumption}}




%\DeclareMathOperator{\dim}{\mathrm{dim}}
\DeclareMathOperator{\Tr}{Tr}
\DeclareMathOperator{\St}{St}
\DeclareMathOperator{\Sp}{\mathrm{Sp}}
\DeclareMathOperator{\Diam}{\mathrm{Diam}}
\DeclareMathOperator{\Diag}{\mathrm{Diag}}
\DeclareMathOperator{\rank}{\mathrm{rk}}
\DeclareMathOperator{\Det}{Det}
\DeclareMathOperator{\Vol}{Vol}
\DeclareMathOperator{\Adj}{Adj}
\DeclareMathOperator{\Span}{\mathrm{Span}}
\DeclareMathOperator{\Conf}{\mathrm{Conf}}
\DeclareMathOperator{\Fr}{\mathrm{Fr}}
\DeclareMathOperator{\DPP}{\mathrm{DPP}}

\DeclareMathOperator{\Cor}{\mathrm{Cor}}
\DeclareMathOperator{\VS}{\mathrm{VS}}
\DeclareMathOperator{\HS}{\mathrm{HS}}
\DeclareMathOperator{\OP}{\mathrm{op}}
\DeclareMathOperator{\eff}{\mathrm{eff}}
\DeclareMathOperator{\Tran}{\intercal}
\newcommand{\dataset}{{\cal D}}
\newcommand{\fracpartial}[2]{\frac{\partial #1}{\partial  #2}}
\DeclareMathOperator{\EX}{\mathbb{E}}
\DeclareMathOperator{\Var}{\mathbb{V}}
\DeclareMathOperator{\Prb}{\mathbb{P}}
\DeclareMathOperator*{\argmax}{arg\,max}
\DeclareMathOperator*{\argmin}{arg\,min}
\DeclareMathOperator*{\KDPP}{\mathfrak{K}}

\DeclareMathOperator{\F}{\mathcal{F}}
\DeclareMathOperator{\X}{\mathcal{X}}
% Expectation symbol

\DeclareMathOperator{\DPh}{\mathrm{DPh}}

\DeclareMathOperator{\Kerspace}{\mathrm{Ker}}
\DeclareMathOperator{\Imspace}{\mathrm{Im}}

\DeclareMathOperator{\Sinmatrix}{\mathcal{S}}
\DeclareMathOperator{\Cosmatrix}{\mathcal{C}}
\DeclareMathOperator{\Tanmatrix}{\mathcal{T}}
\DeclareMathOperator{\LtwoR}{\mathbb{L}_{2}(\mathbb{R})}


\def\rk{\text{rk}}
% Expectation symbol

%\DeclareRobustCommand{\bbone}{\text{\usefont{U}{bbold}{m}{n}1}}
%\DeclareMathOperator{\Ltwo}{\mathbb{L}_{2}(\mathrm{d} \omega)}

\def\Ltwo{\mathbb{L}_{2}(\mathrm{d} \omega)}

\DeclareMathOperator{\Mu}{\mathrm{d}\omega(x)}
\DeclareMathOperator{\MuTen}{\otimes\mathrm{d}\omega(x_{i})}
\DeclareMathOperator{\Ns}{\mathbb{N}^{*}}
\def\UN{\:\mathcal{U}_N}
\def\UNm{\:\mathcal{U}_N^m}
\def\ind{\mathbbm{1}}


\newcommand{\ar}[1]{\textcolor{magenta}{~\algoremark{#1}}}


\newcommand{\ab}[1]{\textcolor{red}{#1}}
\newcommand{\pc}[1]{\textcolor{blue}{#1}}
\newcommand{\rb}[1]{\textcolor{magenta}{#1}}


\usepackage{bbold}


\newcommand*\newprec{\vcenter{\hbox{\includegraphics[width=0.5cm]{img/cssp/macros/newprec.png}}}}

\newcommand*\newtiltedprec{\vcenter{\hbox{\rotatebox{-30}{\includegraphics[width=0.5cm]{img/cssp/macros/newtiltedprec.png}}}}}
% Definitions of handy macros can go here



% Expectation symbol

%\DeclareRobustCommand{\bbone}{\text{\usefont{U}{bbold}{m}{n}1}}


% Heading arguments are {volume}{year}{pages}{date submitted}{date published}{paper id}{author-full-names}



% Short headings should be running head and authors last names

%\firstpageno{1}

\AtBeginEnvironment{subappendices}{%
\chapter*{Appendix}
\addcontentsline{toc}{chapter}{Appendices}
\counterwithin{figure}{section}
\counterwithin{table}{section}
}

\AtEndEnvironment{subappendices}{%
\counterwithout{figure}{section}
\counterwithout{table}{section}
}

\begin{document}

\title{Résumé en français de\\ "A determinantal point process for column subset selection"}

%
%\newcommand*{\titleGM}{
%\makeatletter
%\def\@maketitle{%
\newcommand{\maketitleGM}{
%\pagestyle{empty}
    \begingroup % Create the command for including the title page in the document
    \thispagestyle{empty}
    \newgeometry{textwidth=15cm, top=0cm, bottom=1.5cm}

    \hbox{ % Horizontal box
        \hspace*{0.025\textwidth} % Whitespace to the left of the title page
        %\rule{2pt}{\textheight} % Vertical line
        %\hspace*{0.05\textwidth} % Whitespace between the vertical line and title page text
        \parbox[c]{\textwidth}{ % Paragraph box which restricts text to less than the width of the page
        \vspace*{-2em}
        {Numéro d'ordre: ??}
        {\LARGE \begin{center} Centrale Lille \end{center} }

        \vspace{2em}

        {\huge \begin{center} Thèse de Doctorat \end{center} }
        % {\Large \begin{center} En \end{center} }
        {\Large \begin{center}
        % Spécialité :
        \textbf{En Automatique, Génie informatique, Traitement du signal et image} \end{center} }

        \vspace{2em}

        \begin{center}
        	\Large Par \huge \textsc{Ayoub Belhadji} % Author name
        \end{center}

        \vspace{2em}

        {\Large
        \begin{center}
            Doctorat délivré par Centrale Lille
        \end{center} }

        \vspace{2em}

        {\begin{center}
        	{\noindent\LARGE \textbf{Subspace sampling using determinantal point processes.}}\\[1\baselineskip]
        	{\noindent\large \emph{Echantillonnage des sous-espaces à l'aide des processus ponctuels déterminantaux}}\\[1\baselineskip]
        \end{center}}

        \vspace{2em}

        {\begin{minipage}[b]{0.55\linewidth}
          \hspace{-2.5em}
          % \large Soutenue le 03 novembre 2020 devant le jury d'examen:
        \end{minipage}
        \hfill
        \begin{minipage}[b]{0.49\linewidth}

        \end{minipage}}

        \vspace{0.5em}

        {\hspace{-3em}
        \begin{tabular}{lll}
        % \textit{Président}
        %     &
        %     Pierre-Olivier {\scshape Amblard}
        %     &
        %     Directeur de Recherche CNRS, Université de Grenoble-Alpes
        %     \\
        \textit{Rapporteurs}
            &   Agnès {\scshape Desolneux }
            &   Directrice de Recherche CNRS, ENS Paris-Saclay
            \\
            &   -
            &   -\\
        \textit{Examinateurs}
            &   -
            &   -
            \\
            &   -
            &   -
            \\
            &   -
            &   -
            \\
        \textit{Directeurs de thèse}
            &   Pierre {\scshape Chainais}
            &   Professeur des universités, Centrale Lille
            \\
            &   Rémi {\scshape Bardenet}
            &   Chargé de Recherche CNRS, Université de Lille
        \end{tabular}
        }
        % \end{center}

        \vspace{2em}

        \begin{center}
            {\large Thèse préparée dans le Laboratoire} \\
            {\large \textit{Centre de Recherche en Informatique Signal et Automatique de Lille}\\ Université de Lille, Centrale Lille, CNRS, UMR 9189 - CRIStAL \\
            École Doctorale SPI 072}
        \end{center}

        \vspace{2em}

        % \begin{center}
        %     \includegraphics[height=1.3cm]{images/logo_centrale.png}\hspace{1em}
        %     \includegraphics[height=1.3cm]{images/logo_univ_lille.pdf}\hspace{1em}
        %     \includegraphics[height=1.3cm]{images/logo_CRIStAL.jpg}\hspace{1em}
        %     \includegraphics[height=1.3cm]{images/logo_CNRS.jpg}\hspace{1em}
        %     \includegraphics[height=1.3cm]{images/logo_INRIA.jpg}
        % \end{center}
        }
    }
    \endgroup
}
\makeatother

%\maketitleGM
\author{Ayoub Belhadji} %if necessary, replace with your course title
 
\maketitle
%\author{\name Authors}

%\editor{Editors}

%\maketitle

%\begin{abstract}%   <- trailing '%' for backward compatibility of .sty file
%aaa
%\end{abstract}




%\section{Introduction en français}



%\section{Résume CSSP}
La réduction de dimension est une tâche récurrente de l'analyse des signaux en grande dimension. Les axes principaux de l'analyse en composantes principales sont difficilement interprétables. Lorsqu'on souhaite préserver l’interprétabilité des dimensions, la sélection d'attributs est préférable, mais implique a priori une optimisation combinatoire très coûteuse. Cet article contourne la difficulté en proposant un nouvel algorithme de sélection aléatoire d’attributs, ou de manière équivalente, de colonnes de la matrice des données. On utilise pour cela un processus ponctuel déterminantal sur les indices des colonnes, dont le noyau est contrôlé par la structure des données.


\end{document}
